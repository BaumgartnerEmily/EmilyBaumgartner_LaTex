Ce document est un rapport de cours qui se réfère au cours de LaTeX de monsieur Chevallier suivi par @author.
\section{Contexte}
Nous avons suivi des cours de Latex durant la semaine du 29 août au 2 septembre 2022 de 13h15 jusq'à 16h30. Les cours étaient en présentiels et obligatoires pour chaque étudiante/étudiant.
%%if
\section{Citations et bibliographie}
Citer vos sources est essentiel. Avec \texttt{biblatex} vous pouvez facilement citer des articles, des livres ou des sites internet. Toutes les citations dans le texte seront automatiquement regroupées en fin de document dans la section \guillemotleft Bibliographie\guillemotright. Par exemple, citons un article d'Einstein \cite{einstein} ou le livre de Dirac \cite{dirac}.

Parfois il peut être utile d'utiliser un gestionnaire de bibliographie. La communauté académique recommande l'outil \href{https://www.zotero.org/}{Zotero} qui permet de gérer une bibliothèque numérique d'ouvrages et de références numériques. Il permet également de générer une bibliographie compatible avec \LaTeX.

Notez qu'il est très facile d'obtenir l'extrait \texttt{bibtex} depuis des journaux. Sélectionnez \emph{export/citation}. Si vous le pouvez choisissez \texttt{bibtex}. Dans le cas d'un format \texttt{.ris}, utilisez un convertisseur en ligne comme \href{http://www.bruot.org/ris2bib/}{ris2bib}.

\section{Modèle}
Ce document est tiré d'un modèle \LaTeX~ disponible sur Github  et fournis aux étudiantes/étudiants de la Haute Ecole d'Ingénierie de de Gestion du canton de Vaud.
%%fi